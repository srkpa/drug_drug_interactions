
% Default to the notebook output style

    


% Inherit from the specified cell style.




    
\documentclass[11pt]{article}

    
    
    \usepackage[T1]{fontenc}
    % Nicer default font (+ math font) than Computer Modern for most use cases
    \usepackage{mathpazo}

    % Basic figure setup, for now with no caption control since it's done
    % automatically by Pandoc (which extracts ![](path) syntax from Markdown).
    \usepackage{graphicx}
    % We will generate all images so they have a width \maxwidth. This means
    % that they will get their normal width if they fit onto the page, but
    % are scaled down if they would overflow the margins.
    \makeatletter
    \def\maxwidth{\ifdim\Gin@nat@width>\linewidth\linewidth
    \else\Gin@nat@width\fi}
    \makeatother
    \let\Oldincludegraphics\includegraphics
    % Set max figure width to be 80% of text width, for now hardcoded.
    \renewcommand{\includegraphics}[1]{\Oldincludegraphics[width=.8\maxwidth]{#1}}
    % Ensure that by default, figures have no caption (until we provide a
    % proper Figure object with a Caption API and a way to capture that
    % in the conversion process - todo).
    \usepackage{caption}
    \DeclareCaptionLabelFormat{nolabel}{}
    \captionsetup{labelformat=nolabel}

    \usepackage{adjustbox} % Used to constrain images to a maximum size 
    \usepackage{xcolor} % Allow colors to be defined
    \usepackage{enumerate} % Needed for markdown enumerations to work
    \usepackage{geometry} % Used to adjust the document margins
    \usepackage{amsmath} % Equations
    \usepackage{amssymb} % Equations
    \usepackage{textcomp} % defines textquotesingle
    % Hack from http://tex.stackexchange.com/a/47451/13684:
    \AtBeginDocument{%
        \def\PYZsq{\textquotesingle}% Upright quotes in Pygmentized code
    }
    \usepackage{upquote} % Upright quotes for verbatim code
    \usepackage{eurosym} % defines \euro
    \usepackage[mathletters]{ucs} % Extended unicode (utf-8) support
    \usepackage[utf8x]{inputenc} % Allow utf-8 characters in the tex document
    \usepackage{fancyvrb} % verbatim replacement that allows latex
    \usepackage{grffile} % extends the file name processing of package graphics 
                         % to support a larger range 
    % The hyperref package gives us a pdf with properly built
    % internal navigation ('pdf bookmarks' for the table of contents,
    % internal cross-reference links, web links for URLs, etc.)
    \usepackage{hyperref}
    \usepackage{longtable} % longtable support required by pandoc >1.10
    \usepackage{booktabs}  % table support for pandoc > 1.12.2
    \usepackage[inline]{enumitem} % IRkernel/repr support (it uses the enumerate* environment)
    \usepackage[normalem]{ulem} % ulem is needed to support strikethroughs (\sout)
                                % normalem makes italics be italics, not underlines
    \usepackage{mathrsfs}
    

    
    
    % Colors for the hyperref package
    \definecolor{urlcolor}{rgb}{0,.145,.698}
    \definecolor{linkcolor}{rgb}{.71,0.21,0.01}
    \definecolor{citecolor}{rgb}{.12,.54,.11}

    % ANSI colors
    \definecolor{ansi-black}{HTML}{3E424D}
    \definecolor{ansi-black-intense}{HTML}{282C36}
    \definecolor{ansi-red}{HTML}{E75C58}
    \definecolor{ansi-red-intense}{HTML}{B22B31}
    \definecolor{ansi-green}{HTML}{00A250}
    \definecolor{ansi-green-intense}{HTML}{007427}
    \definecolor{ansi-yellow}{HTML}{DDB62B}
    \definecolor{ansi-yellow-intense}{HTML}{B27D12}
    \definecolor{ansi-blue}{HTML}{208FFB}
    \definecolor{ansi-blue-intense}{HTML}{0065CA}
    \definecolor{ansi-magenta}{HTML}{D160C4}
    \definecolor{ansi-magenta-intense}{HTML}{A03196}
    \definecolor{ansi-cyan}{HTML}{60C6C8}
    \definecolor{ansi-cyan-intense}{HTML}{258F8F}
    \definecolor{ansi-white}{HTML}{C5C1B4}
    \definecolor{ansi-white-intense}{HTML}{A1A6B2}
    \definecolor{ansi-default-inverse-fg}{HTML}{FFFFFF}
    \definecolor{ansi-default-inverse-bg}{HTML}{000000}

    % commands and environments needed by pandoc snippets
    % extracted from the output of `pandoc -s`
    \providecommand{\tightlist}{%
      \setlength{\itemsep}{0pt}\setlength{\parskip}{0pt}}
    \DefineVerbatimEnvironment{Highlighting}{Verbatim}{commandchars=\\\{\}}
    % Add ',fontsize=\small' for more characters per line
    \newenvironment{Shaded}{}{}
    \newcommand{\KeywordTok}[1]{\textcolor[rgb]{0.00,0.44,0.13}{\textbf{{#1}}}}
    \newcommand{\DataTypeTok}[1]{\textcolor[rgb]{0.56,0.13,0.00}{{#1}}}
    \newcommand{\DecValTok}[1]{\textcolor[rgb]{0.25,0.63,0.44}{{#1}}}
    \newcommand{\BaseNTok}[1]{\textcolor[rgb]{0.25,0.63,0.44}{{#1}}}
    \newcommand{\FloatTok}[1]{\textcolor[rgb]{0.25,0.63,0.44}{{#1}}}
    \newcommand{\CharTok}[1]{\textcolor[rgb]{0.25,0.44,0.63}{{#1}}}
    \newcommand{\StringTok}[1]{\textcolor[rgb]{0.25,0.44,0.63}{{#1}}}
    \newcommand{\CommentTok}[1]{\textcolor[rgb]{0.38,0.63,0.69}{\textit{{#1}}}}
    \newcommand{\OtherTok}[1]{\textcolor[rgb]{0.00,0.44,0.13}{{#1}}}
    \newcommand{\AlertTok}[1]{\textcolor[rgb]{1.00,0.00,0.00}{\textbf{{#1}}}}
    \newcommand{\FunctionTok}[1]{\textcolor[rgb]{0.02,0.16,0.49}{{#1}}}
    \newcommand{\RegionMarkerTok}[1]{{#1}}
    \newcommand{\ErrorTok}[1]{\textcolor[rgb]{1.00,0.00,0.00}{\textbf{{#1}}}}
    \newcommand{\NormalTok}[1]{{#1}}
    
    % Additional commands for more recent versions of Pandoc
    \newcommand{\ConstantTok}[1]{\textcolor[rgb]{0.53,0.00,0.00}{{#1}}}
    \newcommand{\SpecialCharTok}[1]{\textcolor[rgb]{0.25,0.44,0.63}{{#1}}}
    \newcommand{\VerbatimStringTok}[1]{\textcolor[rgb]{0.25,0.44,0.63}{{#1}}}
    \newcommand{\SpecialStringTok}[1]{\textcolor[rgb]{0.73,0.40,0.53}{{#1}}}
    \newcommand{\ImportTok}[1]{{#1}}
    \newcommand{\DocumentationTok}[1]{\textcolor[rgb]{0.73,0.13,0.13}{\textit{{#1}}}}
    \newcommand{\AnnotationTok}[1]{\textcolor[rgb]{0.38,0.63,0.69}{\textbf{\textit{{#1}}}}}
    \newcommand{\CommentVarTok}[1]{\textcolor[rgb]{0.38,0.63,0.69}{\textbf{\textit{{#1}}}}}
    \newcommand{\VariableTok}[1]{\textcolor[rgb]{0.10,0.09,0.49}{{#1}}}
    \newcommand{\ControlFlowTok}[1]{\textcolor[rgb]{0.00,0.44,0.13}{\textbf{{#1}}}}
    \newcommand{\OperatorTok}[1]{\textcolor[rgb]{0.40,0.40,0.40}{{#1}}}
    \newcommand{\BuiltInTok}[1]{{#1}}
    \newcommand{\ExtensionTok}[1]{{#1}}
    \newcommand{\PreprocessorTok}[1]{\textcolor[rgb]{0.74,0.48,0.00}{{#1}}}
    \newcommand{\AttributeTok}[1]{\textcolor[rgb]{0.49,0.56,0.16}{{#1}}}
    \newcommand{\InformationTok}[1]{\textcolor[rgb]{0.38,0.63,0.69}{\textbf{\textit{{#1}}}}}
    \newcommand{\WarningTok}[1]{\textcolor[rgb]{0.38,0.63,0.69}{\textbf{\textit{{#1}}}}}
    
    
    % Define a nice break command that doesn't care if a line doesn't already
    % exist.
    \def\br{\hspace*{\fill} \\* }
    % Math Jax compatibility definitions
    \def\gt{>}
    \def\lt{<}
    \let\Oldtex\TeX
    \let\Oldlatex\LaTeX
    \renewcommand{\TeX}{\textrm{\Oldtex}}
    \renewcommand{\LaTeX}{\textrm{\Oldlatex}}
    % Document parameters
    % Document title
    \title{Rapport}
    
    
    
    
    

    % Pygments definitions
    
\makeatletter
\def\PY@reset{\let\PY@it=\relax \let\PY@bf=\relax%
    \let\PY@ul=\relax \let\PY@tc=\relax%
    \let\PY@bc=\relax \let\PY@ff=\relax}
\def\PY@tok#1{\csname PY@tok@#1\endcsname}
\def\PY@toks#1+{\ifx\relax#1\empty\else%
    \PY@tok{#1}\expandafter\PY@toks\fi}
\def\PY@do#1{\PY@bc{\PY@tc{\PY@ul{%
    \PY@it{\PY@bf{\PY@ff{#1}}}}}}}
\def\PY#1#2{\PY@reset\PY@toks#1+\relax+\PY@do{#2}}

\expandafter\def\csname PY@tok@w\endcsname{\def\PY@tc##1{\textcolor[rgb]{0.73,0.73,0.73}{##1}}}
\expandafter\def\csname PY@tok@c\endcsname{\let\PY@it=\textit\def\PY@tc##1{\textcolor[rgb]{0.25,0.50,0.50}{##1}}}
\expandafter\def\csname PY@tok@cp\endcsname{\def\PY@tc##1{\textcolor[rgb]{0.74,0.48,0.00}{##1}}}
\expandafter\def\csname PY@tok@k\endcsname{\let\PY@bf=\textbf\def\PY@tc##1{\textcolor[rgb]{0.00,0.50,0.00}{##1}}}
\expandafter\def\csname PY@tok@kp\endcsname{\def\PY@tc##1{\textcolor[rgb]{0.00,0.50,0.00}{##1}}}
\expandafter\def\csname PY@tok@kt\endcsname{\def\PY@tc##1{\textcolor[rgb]{0.69,0.00,0.25}{##1}}}
\expandafter\def\csname PY@tok@o\endcsname{\def\PY@tc##1{\textcolor[rgb]{0.40,0.40,0.40}{##1}}}
\expandafter\def\csname PY@tok@ow\endcsname{\let\PY@bf=\textbf\def\PY@tc##1{\textcolor[rgb]{0.67,0.13,1.00}{##1}}}
\expandafter\def\csname PY@tok@nb\endcsname{\def\PY@tc##1{\textcolor[rgb]{0.00,0.50,0.00}{##1}}}
\expandafter\def\csname PY@tok@nf\endcsname{\def\PY@tc##1{\textcolor[rgb]{0.00,0.00,1.00}{##1}}}
\expandafter\def\csname PY@tok@nc\endcsname{\let\PY@bf=\textbf\def\PY@tc##1{\textcolor[rgb]{0.00,0.00,1.00}{##1}}}
\expandafter\def\csname PY@tok@nn\endcsname{\let\PY@bf=\textbf\def\PY@tc##1{\textcolor[rgb]{0.00,0.00,1.00}{##1}}}
\expandafter\def\csname PY@tok@ne\endcsname{\let\PY@bf=\textbf\def\PY@tc##1{\textcolor[rgb]{0.82,0.25,0.23}{##1}}}
\expandafter\def\csname PY@tok@nv\endcsname{\def\PY@tc##1{\textcolor[rgb]{0.10,0.09,0.49}{##1}}}
\expandafter\def\csname PY@tok@no\endcsname{\def\PY@tc##1{\textcolor[rgb]{0.53,0.00,0.00}{##1}}}
\expandafter\def\csname PY@tok@nl\endcsname{\def\PY@tc##1{\textcolor[rgb]{0.63,0.63,0.00}{##1}}}
\expandafter\def\csname PY@tok@ni\endcsname{\let\PY@bf=\textbf\def\PY@tc##1{\textcolor[rgb]{0.60,0.60,0.60}{##1}}}
\expandafter\def\csname PY@tok@na\endcsname{\def\PY@tc##1{\textcolor[rgb]{0.49,0.56,0.16}{##1}}}
\expandafter\def\csname PY@tok@nt\endcsname{\let\PY@bf=\textbf\def\PY@tc##1{\textcolor[rgb]{0.00,0.50,0.00}{##1}}}
\expandafter\def\csname PY@tok@nd\endcsname{\def\PY@tc##1{\textcolor[rgb]{0.67,0.13,1.00}{##1}}}
\expandafter\def\csname PY@tok@s\endcsname{\def\PY@tc##1{\textcolor[rgb]{0.73,0.13,0.13}{##1}}}
\expandafter\def\csname PY@tok@sd\endcsname{\let\PY@it=\textit\def\PY@tc##1{\textcolor[rgb]{0.73,0.13,0.13}{##1}}}
\expandafter\def\csname PY@tok@si\endcsname{\let\PY@bf=\textbf\def\PY@tc##1{\textcolor[rgb]{0.73,0.40,0.53}{##1}}}
\expandafter\def\csname PY@tok@se\endcsname{\let\PY@bf=\textbf\def\PY@tc##1{\textcolor[rgb]{0.73,0.40,0.13}{##1}}}
\expandafter\def\csname PY@tok@sr\endcsname{\def\PY@tc##1{\textcolor[rgb]{0.73,0.40,0.53}{##1}}}
\expandafter\def\csname PY@tok@ss\endcsname{\def\PY@tc##1{\textcolor[rgb]{0.10,0.09,0.49}{##1}}}
\expandafter\def\csname PY@tok@sx\endcsname{\def\PY@tc##1{\textcolor[rgb]{0.00,0.50,0.00}{##1}}}
\expandafter\def\csname PY@tok@m\endcsname{\def\PY@tc##1{\textcolor[rgb]{0.40,0.40,0.40}{##1}}}
\expandafter\def\csname PY@tok@gh\endcsname{\let\PY@bf=\textbf\def\PY@tc##1{\textcolor[rgb]{0.00,0.00,0.50}{##1}}}
\expandafter\def\csname PY@tok@gu\endcsname{\let\PY@bf=\textbf\def\PY@tc##1{\textcolor[rgb]{0.50,0.00,0.50}{##1}}}
\expandafter\def\csname PY@tok@gd\endcsname{\def\PY@tc##1{\textcolor[rgb]{0.63,0.00,0.00}{##1}}}
\expandafter\def\csname PY@tok@gi\endcsname{\def\PY@tc##1{\textcolor[rgb]{0.00,0.63,0.00}{##1}}}
\expandafter\def\csname PY@tok@gr\endcsname{\def\PY@tc##1{\textcolor[rgb]{1.00,0.00,0.00}{##1}}}
\expandafter\def\csname PY@tok@ge\endcsname{\let\PY@it=\textit}
\expandafter\def\csname PY@tok@gs\endcsname{\let\PY@bf=\textbf}
\expandafter\def\csname PY@tok@gp\endcsname{\let\PY@bf=\textbf\def\PY@tc##1{\textcolor[rgb]{0.00,0.00,0.50}{##1}}}
\expandafter\def\csname PY@tok@go\endcsname{\def\PY@tc##1{\textcolor[rgb]{0.53,0.53,0.53}{##1}}}
\expandafter\def\csname PY@tok@gt\endcsname{\def\PY@tc##1{\textcolor[rgb]{0.00,0.27,0.87}{##1}}}
\expandafter\def\csname PY@tok@err\endcsname{\def\PY@bc##1{\setlength{\fboxsep}{0pt}\fcolorbox[rgb]{1.00,0.00,0.00}{1,1,1}{\strut ##1}}}
\expandafter\def\csname PY@tok@kc\endcsname{\let\PY@bf=\textbf\def\PY@tc##1{\textcolor[rgb]{0.00,0.50,0.00}{##1}}}
\expandafter\def\csname PY@tok@kd\endcsname{\let\PY@bf=\textbf\def\PY@tc##1{\textcolor[rgb]{0.00,0.50,0.00}{##1}}}
\expandafter\def\csname PY@tok@kn\endcsname{\let\PY@bf=\textbf\def\PY@tc##1{\textcolor[rgb]{0.00,0.50,0.00}{##1}}}
\expandafter\def\csname PY@tok@kr\endcsname{\let\PY@bf=\textbf\def\PY@tc##1{\textcolor[rgb]{0.00,0.50,0.00}{##1}}}
\expandafter\def\csname PY@tok@bp\endcsname{\def\PY@tc##1{\textcolor[rgb]{0.00,0.50,0.00}{##1}}}
\expandafter\def\csname PY@tok@fm\endcsname{\def\PY@tc##1{\textcolor[rgb]{0.00,0.00,1.00}{##1}}}
\expandafter\def\csname PY@tok@vc\endcsname{\def\PY@tc##1{\textcolor[rgb]{0.10,0.09,0.49}{##1}}}
\expandafter\def\csname PY@tok@vg\endcsname{\def\PY@tc##1{\textcolor[rgb]{0.10,0.09,0.49}{##1}}}
\expandafter\def\csname PY@tok@vi\endcsname{\def\PY@tc##1{\textcolor[rgb]{0.10,0.09,0.49}{##1}}}
\expandafter\def\csname PY@tok@vm\endcsname{\def\PY@tc##1{\textcolor[rgb]{0.10,0.09,0.49}{##1}}}
\expandafter\def\csname PY@tok@sa\endcsname{\def\PY@tc##1{\textcolor[rgb]{0.73,0.13,0.13}{##1}}}
\expandafter\def\csname PY@tok@sb\endcsname{\def\PY@tc##1{\textcolor[rgb]{0.73,0.13,0.13}{##1}}}
\expandafter\def\csname PY@tok@sc\endcsname{\def\PY@tc##1{\textcolor[rgb]{0.73,0.13,0.13}{##1}}}
\expandafter\def\csname PY@tok@dl\endcsname{\def\PY@tc##1{\textcolor[rgb]{0.73,0.13,0.13}{##1}}}
\expandafter\def\csname PY@tok@s2\endcsname{\def\PY@tc##1{\textcolor[rgb]{0.73,0.13,0.13}{##1}}}
\expandafter\def\csname PY@tok@sh\endcsname{\def\PY@tc##1{\textcolor[rgb]{0.73,0.13,0.13}{##1}}}
\expandafter\def\csname PY@tok@s1\endcsname{\def\PY@tc##1{\textcolor[rgb]{0.73,0.13,0.13}{##1}}}
\expandafter\def\csname PY@tok@mb\endcsname{\def\PY@tc##1{\textcolor[rgb]{0.40,0.40,0.40}{##1}}}
\expandafter\def\csname PY@tok@mf\endcsname{\def\PY@tc##1{\textcolor[rgb]{0.40,0.40,0.40}{##1}}}
\expandafter\def\csname PY@tok@mh\endcsname{\def\PY@tc##1{\textcolor[rgb]{0.40,0.40,0.40}{##1}}}
\expandafter\def\csname PY@tok@mi\endcsname{\def\PY@tc##1{\textcolor[rgb]{0.40,0.40,0.40}{##1}}}
\expandafter\def\csname PY@tok@il\endcsname{\def\PY@tc##1{\textcolor[rgb]{0.40,0.40,0.40}{##1}}}
\expandafter\def\csname PY@tok@mo\endcsname{\def\PY@tc##1{\textcolor[rgb]{0.40,0.40,0.40}{##1}}}
\expandafter\def\csname PY@tok@ch\endcsname{\let\PY@it=\textit\def\PY@tc##1{\textcolor[rgb]{0.25,0.50,0.50}{##1}}}
\expandafter\def\csname PY@tok@cm\endcsname{\let\PY@it=\textit\def\PY@tc##1{\textcolor[rgb]{0.25,0.50,0.50}{##1}}}
\expandafter\def\csname PY@tok@cpf\endcsname{\let\PY@it=\textit\def\PY@tc##1{\textcolor[rgb]{0.25,0.50,0.50}{##1}}}
\expandafter\def\csname PY@tok@c1\endcsname{\let\PY@it=\textit\def\PY@tc##1{\textcolor[rgb]{0.25,0.50,0.50}{##1}}}
\expandafter\def\csname PY@tok@cs\endcsname{\let\PY@it=\textit\def\PY@tc##1{\textcolor[rgb]{0.25,0.50,0.50}{##1}}}

\def\PYZbs{\char`\\}
\def\PYZus{\char`\_}
\def\PYZob{\char`\{}
\def\PYZcb{\char`\}}
\def\PYZca{\char`\^}
\def\PYZam{\char`\&}
\def\PYZlt{\char`\<}
\def\PYZgt{\char`\>}
\def\PYZsh{\char`\#}
\def\PYZpc{\char`\%}
\def\PYZdl{\char`\$}
\def\PYZhy{\char`\-}
\def\PYZsq{\char`\'}
\def\PYZdq{\char`\"}
\def\PYZti{\char`\~}
% for compatibility with earlier versions
\def\PYZat{@}
\def\PYZlb{[}
\def\PYZrb{]}
\makeatother


    % Exact colors from NB
    \definecolor{incolor}{rgb}{0.0, 0.0, 0.5}
    \definecolor{outcolor}{rgb}{0.545, 0.0, 0.0}



    
    % Prevent overflowing lines due to hard-to-break entities
    \sloppy 
    % Setup hyperref package
    \hypersetup{
      breaklinks=true,  % so long urls are correctly broken across lines
      colorlinks=true,
      urlcolor=urlcolor,
      linkcolor=linkcolor,
      citecolor=citecolor,
      }
    % Slightly bigger margins than the latex defaults
    
    \geometry{verbose,tmargin=1in,bmargin=1in,lmargin=1in,rmargin=1in}
    
    

    \begin{document}
    
    
    \maketitle
    
    

    
    \begin{Verbatim}[commandchars=\\\{\}]
{\color{incolor}In [{\color{incolor}1}]:} \PY{k+kn}{from} \PY{n+nn}{IPython}\PY{n+nn}{.}\PY{n+nn}{display} \PY{k}{import} \PY{n}{Image}\PY{p}{,} \PY{n}{HTML}\PY{p}{,} \PY{n}{display}
        \PY{k+kn}{from} \PY{n+nn}{external\PYZus{}utils}\PY{n+nn}{.}\PY{n+nn}{graphics} \PY{k}{import} \PY{o}{*}
        \PY{k+kn}{from} \PY{n+nn}{external\PYZus{}utils}\PY{n+nn}{.}\PY{n+nn}{load\PYZus{}and\PYZus{}test} \PY{k}{import} \PY{o}{*}
        \PY{o}{\PYZpc{}}\PY{k}{matplotlib} inline
        \PY{k+kn}{from} \PY{n+nn}{glob} \PY{k}{import} \PY{n}{glob}
        \PY{k+kn}{from} \PY{n+nn}{datasets}\PY{n+nn}{.}\PY{n+nn}{dataset} \PY{k}{import} \PY{o}{*}
\end{Verbatim}

    \subsection{Train test split original -
DeepDDI}\label{train-test-split-original---deepddi}

    Dans ce rapport, nous utilisons le split fourni par l'équipe de Ryu pour
reproduire leurs résultats puis nous comparer au modèle de l'article.

    \subsubsection{1. Description des jeux de
données}\label{description-des-jeux-de-donnuxe9es}

    \paragraph{1.1 Sommaire- Train test split
original}\label{sommaire--train-test-split-original}

    \begin{Verbatim}[commandchars=\\\{\}]
{\color{incolor}In [{\color{incolor}2}]:} \PY{n}{train}\PY{p}{,} \PY{n}{test}\PY{p}{,} \PY{n}{valid}\PY{p}{,} \PY{n}{train\PYZus{}test}\PY{p}{,} \PY{n}{train\PYZus{}valid}\PY{p}{,} \PY{n}{test\PYZus{}valid} \PY{o}{=} \PY{n}{deepddi\PYZus{}initial\PYZus{}train}\PY{p}{(}\PY{l+s+s2}{\PYZdq{}}\PY{l+s+s2}{./data/deepddi/drugbank}\PY{l+s+s2}{\PYZdq{}}\PY{p}{)}
\end{Verbatim}

    \begin{Verbatim}[commandchars=\\\{\}]
Quick Summary \#1.
 There are:
- 192284 interactions
- 86 side effects
- 191878 pairs of drugs
- 115305 pairs of drugs and  115446 interactions for train 
- 38405 pairs of drugs and  38419 interactions for test
- 38399 pairs of drug and  38419 interactions for valid
- 141 pairs of drugs with 2 type of ddi in train subset only.
- 20 pairs of drugs with 2 type of ddi in valid subset only.
- 14 pairs of drugs with 2 type of ddi in test subset only.

Quick Summary \#2
 There are:
- 114 pairs of drugs are present in train and test subsets.
- 93 pairs of drugs are present in train and valid subsets.
- 24 pairs of drugs are present in test and valid subsets.
- 0 pairs of drugs are present in train, test and valid subsets.

Quick Summary \#3:
- pairs of drugs that belong to two of the three subsets have always two types of ddi which have been split into the subsets
- pairs of drugs that belong only to one of the three subsets, can have 1 or 2 types of ddi, but all the type will belong to the choosen subset

    \end{Verbatim}

    \paragraph{1.2 Analyse des ensembles d'entrainement, de validation et de
test}\label{analyse-des-ensembles-dentrainement-de-validation-et-de-test}

    \begin{itemize}
\item
  Le tableau qui suit indique décrit les 114 paires de médicaments qui
  apparaissent à la fois dans le train et le test.
\item
  Les chiffres incrits sous les colonnes train et test sont des types
  d'effets secondaires.
\end{itemize}

    \begin{Verbatim}[commandchars=\\\{\}]
{\color{incolor}In [{\color{incolor}3}]:} \PY{n}{df} \PY{o}{=} \PY{n}{pd}\PY{o}{.}\PY{n}{DataFrame}\PY{o}{.}\PY{n}{from\PYZus{}records}\PY{p}{(}\PY{n}{train\PYZus{}test}\PY{p}{,} \PY{n}{columns}\PY{o}{=}\PY{p}{[}\PY{l+s+s2}{\PYZdq{}}\PY{l+s+s2}{pairs of drugs}\PY{l+s+s2}{\PYZdq{}}\PY{p}{,} \PY{l+s+s2}{\PYZdq{}}\PY{l+s+s2}{train}\PY{l+s+s2}{\PYZdq{}}\PY{p}{,} \PY{l+s+s2}{\PYZdq{}}\PY{l+s+s2}{test}\PY{l+s+s2}{\PYZdq{}}\PY{p}{]}\PY{p}{)}
        \PY{n}{display}\PY{p}{(}\PY{n}{df}\PY{p}{)}
\end{Verbatim}

    
    \begin{verbatim}
         pairs of drugs train test
0    (DB00254, DB00997)     6   10
1    (DB00401, DB01601)     6   10
2    (DB00997, DB01118)    10    6
3    (DB00836, DB00997)    10    6
4    (DB00997, DB01104)    10    6
5    (DB00420, DB00834)    76   10
6    (DB00938, DB01072)     6   10
7    (DB00705, DB00997)    10    6
8    (DB00401, DB01232)    10    6
9    (DB00541, DB12332)     6   10
10   (DB00938, DB09065)    10    6
11   (DB00285, DB00541)     6   10
12   (DB00661, DB01219)    61    6
13   (DB00215, DB01100)    32    6
14   (DB00715, DB00997)    10    6
15   (DB00752, DB00956)    54   26
16   (DB00834, DB08903)    76   10
17   (DB00243, DB09280)    10    9
18   (DB00752, DB00921)    26   54
19   (DB00334, DB00349)    26    6
20   (DB00624, DB00997)    10    6
21   (DB00224, DB00997)    10    6
22   (DB00343, DB00997)    10    6
23   (DB00938, DB01264)    10    6
24   (DB00243, DB01174)     9   10
25   (DB00564, DB06414)     9    7
26   (DB00752, DB00924)    79   26
27   (DB00834, DB09078)    10   76
28   (DB00794, DB06708)     7    9
29   (DB00541, DB01076)    26   10
..                  ...   ...  ...
84   (DB00220, DB00997)    10    6
85   (DB00613, DB01029)    10    6
86   (DB00997, DB04855)     6   10
87   (DB00188, DB00997)    10    6
88   (DB00997, DB01221)    10    6
89   (DB00206, DB09027)     9   10
90   (DB00401, DB01201)     7    9
91   (DB00390, DB00541)    17    9
92   (DB01320, DB06697)     9   12
93   (DB00938, DB01263)    10    6
94   (DB00608, DB00997)    10   15
95   (DB00683, DB09027)    10    9
96   (DB00238, DB06697)    12    9
97   (DB00834, DB01182)    76   10
98   (DB00182, DB00921)    26   44
99   (DB00477, DB00997)    10    6
100  (DB00683, DB00997)    10    6
101  (DB00541, DB08865)     6   10
102  (DB00196, DB00541)    10    6
103  (DB00196, DB00997)     6   10
104  (DB00541, DB00834)    10    9
105  (DB00363, DB00997)    10   26
106  (DB00396, DB09027)     9   10
107  (DB00312, DB06708)     7    9
108  (DB01115, DB09027)    10    9
109  (DB00501, DB00884)     2   10
110  (DB00401, DB01118)    10    6
111  (DB00327, DB00752)     6   26
112  (DB00541, DB00615)     7    9
113  (DB00541, DB08873)    10    6

[114 rows x 3 columns]
    \end{verbatim}

    
    \begin{itemize}
\item
  Le tableau qui suit indique décrit les 93 paires de médicaments qui
  apparaissent à la fois dans le train et le valid.
\item
  Les chiffres incrits sous les colonnes train et valid sont des types
  d'effets secondaires.
\end{itemize}

    \begin{Verbatim}[commandchars=\\\{\}]
{\color{incolor}In [{\color{incolor}4}]:} \PY{n}{df} \PY{o}{=} \PY{n}{pd}\PY{o}{.}\PY{n}{DataFrame}\PY{o}{.}\PY{n}{from\PYZus{}records}\PY{p}{(}\PY{n}{train\PYZus{}valid}\PY{p}{,} \PY{n}{columns}\PY{o}{=}\PY{p}{[}\PY{l+s+s2}{\PYZdq{}}\PY{l+s+s2}{pairs of drugs}\PY{l+s+s2}{\PYZdq{}}\PY{p}{,} \PY{l+s+s2}{\PYZdq{}}\PY{l+s+s2}{train}\PY{l+s+s2}{\PYZdq{}}\PY{p}{,} \PY{l+s+s2}{\PYZdq{}}\PY{l+s+s2}{valid}\PY{l+s+s2}{\PYZdq{}}\PY{p}{]}\PY{p}{)}
        \PY{n}{display}\PY{p}{(}\PY{n}{df}\PY{p}{)}
\end{Verbatim}

    
    \begin{verbatim}
        pairs of drugs train valid
0   (DB00220, DB00541)     6    10
1   (DB00238, DB06708)     7     9
2   (DB00263, DB00997)    10     6
3   (DB00613, DB00675)    10     6
4   (DB00997, DB01072)    10     6
5   (DB00613, DB08880)    10     6
6   (DB00257, DB00541)     6     9
7   (DB00997, DB04871)    10     6
8   (DB00997, DB01076)    10    26
9   (DB00312, DB00541)     9     7
10  (DB00476, DB00997)    10     6
11  (DB00541, DB01026)    10     6
12  (DB00220, DB00401)    10     6
13  (DB00541, DB01263)    26    10
14  (DB01201, DB09123)     7     9
15  (DB00401, DB00794)     7     9
16  (DB00997, DB01403)    10     6
17  (DB01201, DB06708)     9     7
18  (DB00997, DB09054)    10     6
19  (DB00541, DB01200)    26    10
20  (DB00541, DB01253)    26    10
21  (DB00834, DB09027)    10     9
22  (DB00422, DB01626)    55    35
23  (DB00224, DB09027)    10     9
24  (DB00997, DB01167)    10     6
25  (DB01224, DB01233)    76    26
26  (DB00238, DB00541)     9     7
27  (DB00694, DB09027)     9    10
28  (DB01045, DB09027)     9    10
29  (DB00503, DB00541)     9    10
..                 ...   ...   ...
63  (DB00997, DB01200)    26    10
64  (DB01174, DB09123)    13     9
65  (DB00997, DB01026)    10     6
66  (DB00091, DB00541)     9     6
67  (DB00924, DB01367)    26    79
68  (DB00349, DB00997)    10     6
69  (DB00541, DB01045)     7     9
70  (DB00582, DB00938)     6    10
71  (DB00951, DB00997)     6    10
72  (DB00613, DB00625)    20    10
73  (DB00312, DB00997)     7     9
74  (DB00208, DB00997)    10     6
75  (DB00794, DB01115)     9     7
76  (DB00924, DB01037)    79    26
77  (DB00875, DB01233)    76    26
78  (DB00613, DB01132)     6    10
79  (DB00997, DB08873)    10     6
80  (DB00564, DB06697)     7    12
81  (DB01233, DB01624)    76    26
82  (DB00401, DB01167)     6    10
83  (DB00195, DB00997)     6    10
84  (DB00401, DB01263)     6    10
85  (DB00613, DB01685)     6    10
86  (DB00220, DB09027)     9    10
87  (DB00338, DB00884)    10    13
88  (DB00997, DB08865)    10     6
89  (DB00514, DB00997)     6    10
90  (DB00701, DB09027)    10     9
91  (DB01156, DB01171)     6    26
92  (DB00541, DB00864)    26     9

[93 rows x 3 columns]
    \end{verbatim}

    
    \begin{itemize}
\item
  Le tableau qui suit indique décrit les 24 paires de médicaments qui
  apparaissent à la fois dans le test et le valid.
\item
  Les chiffres incrits sous les colonnes test et valid sont des types
  d'effets secondaires.
\end{itemize}

    \begin{Verbatim}[commandchars=\\\{\}]
{\color{incolor}In [{\color{incolor}5}]:} \PY{n}{df} \PY{o}{=} \PY{n}{pd}\PY{o}{.}\PY{n}{DataFrame}\PY{o}{.}\PY{n}{from\PYZus{}records}\PY{p}{(}\PY{n}{test\PYZus{}valid}\PY{p}{,} \PY{n}{columns}\PY{o}{=}\PY{p}{[}\PY{l+s+s2}{\PYZdq{}}\PY{l+s+s2}{pairs of drugs}\PY{l+s+s2}{\PYZdq{}}\PY{p}{,} \PY{l+s+s2}{\PYZdq{}}\PY{l+s+s2}{train}\PY{l+s+s2}{\PYZdq{}}\PY{p}{,} \PY{l+s+s2}{\PYZdq{}}\PY{l+s+s2}{valid}\PY{l+s+s2}{\PYZdq{}}\PY{p}{]}\PY{p}{)}
        \PY{n}{display}\PY{p}{(}\PY{n}{df}\PY{p}{)}
\end{Verbatim}

    
    \begin{verbatim}
        pairs of drugs train valid
0   (DB01234, DB09027)     9    10
1   (DB00482, DB00613)     6    10
2   (DB00363, DB01233)    76    26
3   (DB00877, DB09027)    10     9
4   (DB00169, DB00997)     6    10
5   (DB00997, DB01320)     9     7
6   (DB00997, DB11613)    10     6
7   (DB01118, DB09027)    10     9
8   (DB00541, DB00976)    10     6
9   (DB09027, DB09280)     9    10
10  (DB01045, DB06708)     7     9
11  (DB00541, DB00696)    26    10
12  (DB00932, DB00997)     6    10
13  (DB00182, DB00327)    26    44
14  (DB00564, DB09123)     9    13
15  (DB00613, DB01015)     6    10
16  (DB00390, DB09027)    10     9
17  (DB00997, DB04868)     6    10
18  (DB00834, DB00976)    76    10
19  (DB00834, DB01232)    10    76
20  (DB00401, DB01026)    10     6
21  (DB01115, DB01174)     7     9
22  (DB00270, DB00997)     6    10
23  (DB00091, DB09027)     9    10
    \end{verbatim}

    
    \begin{itemize}
\tightlist
\item
  Comme on peut le remarquer, le spit est fait au niveau des types
  d'effets secondaires, pas au niveau des paires de médicaments.
\end{itemize}

    Les tableaux et figures qui suivent décrivent la répartition des labels
dans le dataset final.

    \begin{Verbatim}[commandchars=\\\{\}]
{\color{incolor}In [{\color{incolor}6}]:} \PY{n}{x\PYZus{}train}\PY{p}{,} \PY{n}{x\PYZus{}test}\PY{p}{,} \PY{n}{x\PYZus{}valid}\PY{p}{,} \PY{n}{y\PYZus{}train}\PY{p}{,} \PY{n}{y\PYZus{}test}\PY{p}{,} \PY{n}{y\PYZus{}valid} \PY{o}{=} \PY{n}{load}\PY{p}{(}\PY{n}{train}\PY{p}{,} \PY{n}{test}\PY{p}{,} \PY{n}{valid}\PY{p}{,} \PY{l+m+mi}{86}\PY{p}{,} \PY{n}{method}\PY{o}{=}\PY{l+s+s2}{\PYZdq{}}\PY{l+s+s2}{deepddi}\PY{l+s+s2}{\PYZdq{}}\PY{p}{,} \PY{n}{input\PYZus{}path}\PY{o}{=}\PY{l+s+s2}{\PYZdq{}}\PY{l+s+s2}{./data/deepddi/drugbank}\PY{l+s+s2}{\PYZdq{}}\PY{p}{)}
\end{Verbatim}

    \begin{Verbatim}[commandchars=\\\{\}]
/home/rogia/Documents/code/side\_effects/feature\_extraction/transformers/molecules.py:521: RuntimeWarning: Some out of vocabulary encounter during the transformation.Please make sure you vocabulary is exhaustive.
  'Please make sure you vocabulary is exhaustive.', RuntimeWarning)

    \end{Verbatim}

    \begin{Verbatim}[commandchars=\\\{\}]
Quick summary \#4
 Final subsets:

- 115266 pairs of drugs for train
- 38387 pairs of drugs for test
- 38389 pairs of drug for valid

    \end{Verbatim}

    \begin{Verbatim}[commandchars=\\\{\}]
{\color{incolor}In [{\color{incolor}7}]:} \PY{n}{df} \PY{o}{=} \PY{n}{describe\PYZus{}data}\PY{p}{(}\PY{n}{y\PYZus{}train}\PY{p}{,} \PY{n}{y\PYZus{}test}\PY{p}{,} \PY{n}{y\PYZus{}valid}\PY{p}{)}
\end{Verbatim}

    \begin{center}
    \adjustimage{max size={0.9\linewidth}{0.9\paperheight}}{output_16_0.png}
    \end{center}
    { \hspace*{\fill} \\}
    
    \begin{Verbatim}[commandchars=\\\{\}]
{\color{incolor}In [{\color{incolor}8}]:} \PY{n}{display}\PY{p}{(}\PY{n}{df}\PY{p}{)}
\end{Verbatim}

    
    \begin{verbatim}
            1     2     3      4     5        6       7     8       9  \
train   634.0  50.0  27.0  313.0   7.0  20616.0  3007.0   7.0  5687.0   
test    207.0  16.0   9.0  103.0   2.0   6872.0  1002.0   2.0  1895.0   
valid   214.0  16.0   9.0  103.0   2.0   6872.0  1002.0   2.0  1895.0   
Total  1055.0  82.0  45.0  519.0  11.0  34360.0  5011.0  11.0  9477.0   

            10  ...     77      78     79    80     81     82    83     84  \
train  14269.0  ...  180.0   609.0  483.0  34.0  224.0   66.0  27.0   67.0   
test    4755.0  ...   60.0   203.0  160.0  11.0   74.0   21.0   8.0   21.0   
valid   4755.0  ...   60.0   203.0  160.0  11.0   74.0   21.0   8.0   21.0   
Total  23779.0  ...  300.0  1015.0  803.0  56.0  372.0  108.0  43.0  109.0   

         85    86  
train  21.0  39.0  
test    6.0  13.0  
valid   6.0  13.0  
Total  33.0  65.0  

[4 rows x 86 columns]
    \end{verbatim}

    
    \subsubsection{2. Analyse de l'évolution de la perte en entrainement et
en validation et des performances en
test}\label{analyse-de-luxe9volution-de-la-perte-en-entrainement-et-en-validation-et-des-performances-en-test}

    \subsection{Train test suggéré -
DeepDDI}\label{train-test-sugguxe9ruxe9---deepddi}

    \begin{itemize}
\tightlist
\item
  Dans ce split, la séparation est faite au niveau des paires. On
  s'assure à ce que:

  \begin{itemize}
  \tightlist
  \item
    Les paires de médicaments présents dans les ensembles
    d'entrainement, de validation et de test soient distincts
  \item
    Seulement un des médicaments des paires dans le test et le valid
    aient été vu à l'entrainement avec d,autres médicaments.
  \end{itemize}
\end{itemize}

    \paragraph{2.1 Analyse des ensembles d'entrainement, de validation et de
test}\label{analyse-des-ensembles-dentrainement-de-validation-et-de-test}

    \begin{Verbatim}[commandchars=\\\{\}]
{\color{incolor}In [{\color{incolor}9}]:} \PY{n}{drugbk}\PY{p}{,} \PY{n}{dbk\PYZus{}labels} \PY{o}{=} \PY{n}{describe\PYZus{}dataset}\PY{p}{(}\PY{l+s+s2}{\PYZdq{}}\PY{l+s+s2}{./data/violette/drugbank}\PY{l+s+s2}{\PYZdq{}}\PY{p}{,} \PY{l+s+s2}{\PYZdq{}}\PY{l+s+s2}{drugbank}\PY{l+s+s2}{\PYZdq{}}\PY{p}{)}
\end{Verbatim}

    \begin{center}
    \adjustimage{max size={0.9\linewidth}{0.9\paperheight}}{output_22_0.png}
    \end{center}
    { \hspace*{\fill} \\}
    
    \begin{Verbatim}[commandchars=\\\{\}]
/home/rogia/Documents/code/side\_effects/feature\_extraction/transformers/molecules.py:521: RuntimeWarning: Some out of vocabulary encounter during the transformation.Please make sure you vocabulary is exhaustive.
  'Please make sure you vocabulary is exhaustive.', RuntimeWarning)

    \end{Verbatim}

    \begin{Verbatim}[commandchars=\\\{\}]
len train 119336
len test 26044
len valid 37961
There are 0 intersections entre train et test
There are 0 intersections entre train et valid
There are 0 intersections entre train et test

    \end{Verbatim}

    \begin{center}
    \adjustimage{max size={0.9\linewidth}{0.9\paperheight}}{output_22_3.png}
    \end{center}
    { \hspace*{\fill} \\}
    
    \begin{Verbatim}[commandchars=\\\{\}]
{\color{incolor}In [{\color{incolor}10}]:} \PY{n}{cols} \PY{o}{=} \PY{p}{[}\PY{l+s+s2}{\PYZdq{}}\PY{l+s+s2}{Number of drugs}\PY{l+s+s2}{\PYZdq{}}\PY{p}{,} \PY{l+s+s2}{\PYZdq{}}\PY{l+s+s2}{Number of samples}\PY{l+s+s2}{\PYZdq{}}\PY{p}{,} \PY{l+s+s2}{\PYZdq{}}\PY{l+s+s2}{Drug with the longest smiles}\PY{l+s+s2}{\PYZdq{}}\PY{p}{,} \PY{l+s+s2}{\PYZdq{}}\PY{l+s+s2}{Longest smile length}\PY{l+s+s2}{\PYZdq{}}\PY{p}{,}\PY{l+s+s2}{\PYZdq{}}\PY{l+s+s2}{Number of side effects}\PY{l+s+s2}{\PYZdq{}}\PY{p}{,} \PY{l+s+s2}{\PYZdq{}}\PY{l+s+s2}{train}\PY{l+s+s2}{\PYZdq{}}\PY{p}{,} \PY{l+s+s2}{\PYZdq{}}\PY{l+s+s2}{valid}\PY{l+s+s2}{\PYZdq{}}\PY{p}{,} \PY{l+s+s2}{\PYZdq{}}\PY{l+s+s2}{test}\PY{l+s+s2}{\PYZdq{}}\PY{p}{]}
         \PY{n}{df} \PY{o}{=} \PY{n}{pd}\PY{o}{.}\PY{n}{DataFrame}\PY{o}{.}\PY{n}{from\PYZus{}records}\PY{p}{(}\PY{p}{[}\PY{n}{drugbk}\PY{p}{]}\PY{p}{,} \PY{n}{columns}\PY{o}{=}\PY{n}{cols}\PY{p}{,} \PY{n}{index}\PY{o}{=}\PY{p}{[}\PY{l+s+s2}{\PYZdq{}}\PY{l+s+s2}{Drugbank}\PY{l+s+s2}{\PYZdq{}}\PY{p}{]}\PY{p}{)}
         \PY{n}{display}\PY{p}{(}\PY{n}{df}\PY{p}{)}
\end{Verbatim}

    
    \begin{verbatim}
          Number of drugs  Number of samples Drug with the longest smiles  \
Drugbank             1709             183341                      DB05528   

          Longest smile length  Number of side effects   train   test  valid  
Drugbank                   850                      86  119336  37961  26044  
    \end{verbatim}

    
    \begin{Verbatim}[commandchars=\\\{\}]
{\color{incolor}In [{\color{incolor}11}]:} \PY{n}{display}\PY{p}{(}\PY{n}{dbk\PYZus{}labels}\PY{p}{)}
\end{Verbatim}

    
    \begin{verbatim}
           1       10     11     12      13     14      15    16      17  \
train  519.0  15106.0  404.0  224.0  4690.0  187.0  1153.0   3.0   801.0   
test   259.0   3751.0   60.0   26.0  1243.0   20.0   231.0   0.0   113.0   
valid  185.0   3957.0   72.0   56.0  1484.0   38.0   365.0  20.0   129.0   
Total  963.0  22814.0  536.0  306.0  7417.0  245.0  1749.0  23.0  1043.0   

          18  ...     79    8    80     81     82    83     84    85    86  \
train  154.0  ...  571.0  2.0  31.0  236.0   82.0   0.0   69.0  27.0  32.0   
test    22.0  ...   60.0  7.0   9.0   42.0   17.0  34.0    4.0   6.0   4.0   
valid   26.0  ...  156.0  0.0  14.0   82.0    9.0   0.0   35.0   0.0  24.0   
Total  202.0  ...  787.0  9.0  54.0  360.0  108.0  34.0  108.0  33.0  60.0   

            9  
train  5879.0  
test   1159.0  
valid  2027.0  
Total  9065.0  

[4 rows x 86 columns]
    \end{verbatim}

    
    \paragraph{2.1 Analyse de l'évolution de la perte en entrainement et en
validation et des performances en
test}\label{analyse-de-luxe9volution-de-la-perte-en-entrainement-et-en-validation-et-des-performances-en-test}

    Resumé:

Ici, nous testons DeepDDI et notre modèle sur le nouveau train test
split. Les hyperparamètres utilisés pour DeepDDI sont ceux de l'article.
Pour notre modèle, seule la meilleure combinaison sera retenue.

    \paragraph{2.1.1 DeepDDI}\label{deepddi}

    \begin{Verbatim}[commandchars=\\\{\}]
{\color{incolor}In [{\color{incolor}9}]:} \PY{n}{met1} \PY{o}{=} \PY{n}{describe\PYZus{}expt}\PY{p}{(}\PY{n+nb}{dir}\PY{o}{=}\PY{l+s+s2}{\PYZdq{}}\PY{l+s+s2}{expts/deepdddi\PYZhy{}new\PYZus{}drugbank\PYZus{}init}\PY{l+s+s2}{\PYZdq{}}\PY{p}{)}
        \PY{n}{display}\PY{p}{(}\PY{n}{met1}\PY{p}{)}
\end{Verbatim}

    \begin{center}
    \adjustimage{max size={0.9\linewidth}{0.9\paperheight}}{output_28_0.png}
    \end{center}
    { \hspace*{\fill} \\}
    
    
    \begin{verbatim}
     mean_acc  micro_f1  macro_f1  macro_prc  micro_prc  micro_rec  macro_rec
0.5  0.988856   0.41424  0.154039   0.244687   0.535039   0.337941   0.126998
    \end{verbatim}

    
    \paragraph{2.1.2 Notre modèle}\label{notre-moduxe8le}

    \begin{Verbatim}[commandchars=\\\{\}]
{\color{incolor}In [{\color{incolor}13}]:} \PY{n}{met2} \PY{o}{=} \PY{n}{describe\PYZus{}expt}\PY{p}{(}\PY{n+nb}{dir}\PY{o}{=}\PY{l+s+s2}{\PYZdq{}}\PY{l+s+s2}{expts/ours\PYZus{}new\PYZus{}drugb\PYZus{}emb}\PY{l+s+s2}{\PYZdq{}}\PY{p}{)}
         \PY{n}{display}\PY{p}{(}\PY{n}{met2}\PY{p}{)}
\end{Verbatim}

    \begin{center}
    \adjustimage{max size={0.9\linewidth}{0.9\paperheight}}{output_30_0.png}
    \end{center}
    { \hspace*{\fill} \\}
    
    \begin{center}
    \adjustimage{max size={0.9\linewidth}{0.9\paperheight}}{output_30_1.png}
    \end{center}
    { \hspace*{\fill} \\}
    
    
    \begin{verbatim}
      mean_acc  micro_f1  macro_f1  macro_prc  micro_prc  micro_rec  macro_rec
0.38   0.99106  0.561435  0.122772   0.245889   0.655903   0.490753   0.100161
    \end{verbatim}

    
    \paragraph{2.1.2 Comparaison}\label{comparaison}

    \begin{Verbatim}[commandchars=\\\{\}]
{\color{incolor}In [{\color{incolor}14}]:} \PY{n}{compare}\PY{p}{(}\PY{n}{met1}\PY{p}{,} \PY{n}{met2}\PY{p}{,}\PY{l+s+s2}{\PYZdq{}}\PY{l+s+s2}{deepddi}\PY{l+s+s2}{\PYZdq{}}\PY{p}{,}\PY{l+s+s2}{\PYZdq{}}\PY{l+s+s2}{ours}\PY{l+s+s2}{\PYZdq{}}\PY{p}{)}
\end{Verbatim}

    \begin{center}
    \adjustimage{max size={0.9\linewidth}{0.9\paperheight}}{output_32_0.png}
    \end{center}
    { \hspace*{\fill} \\}
    
    \begin{Verbatim}[commandchars=\\\{\}]
{\color{incolor}In [{\color{incolor} }]:} 
\end{Verbatim}


    % Add a bibliography block to the postdoc
    
    
    
    \end{document}
